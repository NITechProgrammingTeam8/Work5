\documentclass[uplatex,12pt]{jsarticle}
\usepackage[dvipdfmx]{graphicx}
\usepackage{url}
\usepackage{listings,jlisting}
\usepackage{ascmac}
\usepackage{amsmath,amssymb}

%ここからソースコードの表示に関する設定
\lstset{
  basicstyle={\ttfamily},
  identifierstyle={\small},
  commentstyle={\smallitshape},
  keywordstyle={\small\bfseries},
  ndkeywordstyle={\small},
  stringstyle={\small\ttfamily},
  frame={tb},
  breaklines=true,
  columns=[l]{fullflexible},
  numbers=left,
  xrightmargin=0zw,
  xleftmargin=3zw,
  numberstyle={\scriptsize},
  stepnumber=1,
  numbersep=1zw,
  lineskip=-0.5ex
}
%ここまでソースコードの表示に関する設定

\title{知能プログラミング演習II 課題5}
\author{グループ8\\
  29114003 青山周平\\
  29114060 後藤拓也\\
  29114116 増田大輝\\
  29114142 湯浅範子\\
  29119016 小中祐希\\
}
\date{2019年12月10日}

\begin{document}
\maketitle

\paragraph{提出物} group08.pdf, group08.zip
\paragraph{グループ} グループ8

\paragraph{担当}
\begin{tabular}{|c|l|}
  \hline
  氏名&担当課題\\
  \hline\hline
  青山周平&必須課題5-4, 発展課題5-7\\
  \hline
  後藤拓也&必須課題5-2\\
  \hline
  増田大輝&必須課題5-3, 発展課題5-5\\
  \hline
  湯浅範子&必須課題5-4\\
  \hline
  小中祐希&必須課題5-1, 必須課題5-2, 発展課題5-8\\
  \hline
\end{tabular}



\section{課題の説明}
\begin{description}
\item[必須課題5-1] 目標集合を変えてみたときに,動作が正しくない場合があったかどうか,実行例を示して考察せよ.
また,もしあったならその箇所を修正し,どのように修正したか記せ.
\item[必須課題5-2] 教科書のプログラムでは,オペレータ間の競合解消戦略としてランダムなオペレータ選択を採用している.
これを,効果的な競合解消戦略に改良すべく考察し,実装せよ.
改良の結果,性能がどの程度向上したかを定量的に(つまり数字で)示すこと.
\item[必須課題5-3] 上記のプランニングのプログラムでは,ブロックの属性(たとえば色や形など)を考えていないので,色や形などの属性を扱えるようにせよ.ルールとして表現すること.
例えば色と形の両方を扱えるようにする場合,Aが青い三角形,Bが黄色の四角形,Cが緑の台形であったとする.
その時,色と形を使ってもゴールを指定できるようにする("green on blue" や"blue on box"のように)
\item[必須課題5-4] 上記5-2, 5-3で改良したプランニングシステムのGUIを実装せよ.
ブロック操作の過程をグラフィカルに可視化し,初期状態や目標状態をGUI上で変更できることが望ましい.
\item[発展課題5-5] ブロックワールド内における物理的制約条件をルールとして表現せよ.
例えば,三角錐(pyramid)の上には他のブロックを乗せられない等,その世界における物理的な制約を実現せよ.
\item[発展課題5-6] ユーザが自然言語(日本語や英語など)の命令文によってブロックを操作したり,初期状態/目標状態を変更したりできるようにせよ.
なお,命令文の動詞や語尾を1つの表現に決め打ちするのではなく,多様な表現を許容できることが望ましい.
\item[発展課題5-7] 3次元空間 (実世界) の物理的な挙動を考慮したブロックワールドにおけるプランニングを実現せよ.
なお,物理エンジン等を利用する場合,Java以外の言語のフレームワークを使って実現しても構わない.
\item[発展課題5-8] 教科書3.3節のプランニング手法を応用できそうなブロック操作以外のタスクをグループで話し合い,新たなプランニング課題を自由に設定せよ.
さらに,もし可能であれば,その自己設定課題を解くプランニングシステムを実装せよ.
\end{description}

\section{必須課題5-3}
\begin{screen}
    上記のプランニングのプログラムでは,ブロックの属性(たとえば色や形など)を考えていないので,色や形などの属性を扱えるようにせよ.ルールとして表現すること.
    例えば色と形の両方を扱えるようにする場合,Aが青い三角形,Bが黄色の四角形,Cが緑の台形であったとする.
    その時,色と形を使ってもゴールを指定できるようにする("green on blue" や"blue on box"のように)
\end{screen}
\subsection{手法}
ブロックの属性を扱えるようにするために,まずはルールによってブロックと属性の関係を表現し,それらの関係を保持する必要がある.
その上で,属性によって表現されたゴールや,初期状態をブロックによる表現に変換し,後ろ向き推論を行う方針とした. \\
ここで,ブロックの属性によってゴールを表現する場合,そのまま後ろ向き推論を行うと,同一ブロックの異なる属性による表現がワーキングスペースに含まれる場合などへの対応が非常に複雑となる.
したがって,入力されたルールと初期状態はあらかじめ属性表現をブロックに変換することで,もとの後ろ向き推論が正常に機能するような実装を行うものとする.

\subsection{実装}
まず,属性を扱うクラスAttributionsを作成した.
Attributionsクラスの変数としてHashMap<String,String>型のattributions,List<String>型のrulesがある.
それぞれ,属性とブロックの関係(属性が主キー)の集合,ルールの集合に対応する.
このAttributionsクラスには以下の3つのコンストラクタが存在し,全て上記変数への操作を行うものであるが,それぞれに次のような用途がある.

\begin{description}
    \item[Attributions()] テスト用のデフォルトコンストラクタ.デフォルトのルールが用意されている.ルールは自然言語で記述.
    \item[Attributions(List \textless String\textgreater rules)] 自然言語で記述されたルールをもとに属性とブロックの関係を構築する.
    \item[Attributions(HashMap \textless String, String\textgreater attributions)] あらかじめ,属性とブロックの関係が構築された状態のものを受け取る.
\end{description}
ただし,自然言語で記述されているものは,"ブロック is 属性"の形で表現されているものを扱うものとする.
例えば,"A is blue"と記述されたものを受理する. \\

次に,自然言語から属性とブロックの関係を構築するメソッドaddAttribution(String attributionState)についての説明を行う.
このメソッドは上記コンストラクタAttributions()とAttributions(List \textless String\textgreater rules)から呼び出される.
これらのコンストラクタからはrulesの各要素を引数として渡すものとする.
addAttributionメソッドは仮引数attributionStateでこれらを受け取り,"ブロック is 属性"を空白で分割したリストの第二要素が"is"である場合に,attributionsに関係を加える.
この時,属性を主キーとし,ブロックを紐づけられるオブジェクトとする. \\
すなわち,"A is blue"とした場合には,第二要素は"is"であるので"blue"を主キーとして,"A"と紐づけられることとなる. \\

以上により,属性を持つブロックを扱うための下準備が整った. \\

続いて,attributionsに格納された属性とブロックの関係をゴールリストや初期状態のリストに適応するためのメソッドeditStatementList(ArrayList<String> statementList)についての説明である.
このメソッドでは,仮引数として渡されたstatementListの各要素について,attributions内に格納されている属性との照合作業が行われ,属性がある場合には紐づけられたブロック名に置き換えられる作業が行われる.
以下にメソッドのソースコードを示す.
\begin{lstlisting}[caption=editStatementListメソッドの実装, label=mid]
    ArrayList<String> editStatementList(ArrayList<String> statementList) {
		System.out.println("++++++ EditStatement ++++++");
		ArrayList<String> newStatementList = new ArrayList();
		for (String statement: statementList) {
			List<String> tokens = Arrays.asList(statement.split(" "));
			String newStatement = "";
			for(int tokenNum = 0; tokenNum < tokens.size(); tokenNum++) {
				String token = tokens.get(tokenNum);
				if(attributions.containsKey(token)) {
					token = attributions.get(token);
				}
				newStatement += token;
				if(tokenNum < tokens.size()-1) {
					newStatement += " ";
				}
			}
			newStatementList.add(newStatement);
			System.out.println(statement+" =====> "+newStatement);
		}
		return newStatementList;
	}
\end{lstlisting}
特に,9~11行目でHashMapの機能を利用して属性からブロック名を導き出す処理を行っている.

以上によって,自然言語で記述されたルール集合から導出された属性とブロックの関係や与えられた属性とブロックの関係を元にして,
ゴールリストや初期状態のリストに含まれる属性表現をブロック名に統一する処理が完成した.

具体的な利用法は,使用したい任意のコンストラクタによってAttributionsクラスを初期化し,以下のような形で,ゴールリストや初期状態のリストに適用すればよい.
\begin{lstlisting}[caption=editStatementListの利用, label=mid]
    goalList = attributions.editStatementList(initAttributeGoalList());
	initialState = attributions.editStatementList(initAttributeInitialState());
\end{lstlisting}


\subsection{実行例}
以下に,CUI上での実行結果を示す.
ただし,コンストラクタはデフォルトコンストラクタを使用するものとする.
\begin{lstlisting}[caption=ゴールと初期状態に属性を指定した場合の実行結果, label=mid]
    ~/Programming2/Work5
    ●java Planner                                                                                                                                                                                                     【 re-fix/attribute-rule 】
    ========== goal:green on ball ==========
    ========== goal:blue on pyramid ==========
    ++++++ EditStatement ++++++
    green on ball =====> B on C
    blue on pyramid =====> A on B
    ---------- initInitialState:clear blue ----------
    ---------- initInitialState:clear green ----------
    ---------- initInitialState:clear red ----------
    ---------- initInitialState:ontable box ----------
    ---------- initInitialState:ontable pyramid ----------
    ---------- initInitialState:ontable ball ----------
    ---------- initInitialState:handEmpty ----------
    ++++++ EditStatement ++++++
    clear blue =====> clear A
    clear green =====> clear B
    clear red =====> clear C
    ontable box =====> ontable A
    ontable pyramid =====> ontable B
    ontable ball =====> ontable C
    handEmpty =====> handEmpty
    *** GOALS ***[B on C, A on B]
    **B on C
    Place B on C
    *** GOALS ***[clear C, holding B]
    **clear C
    [clear A, clear B, clear C, ontable A, ontable B, ontable C, handEmpty]
    *** GOALS ***[holding B]
    **holding B
    pick up B from the table
    *** GOALS ***[ontable B, clear B, handEmpty]
    **ontable B
    [clear A, clear B, clear C, ontable A, ontable B, ontable C, handEmpty]
    *** GOALS ***[clear B, handEmpty]
    **clear B
    [clear A, clear B, clear C, ontable A, ontable B, ontable C, handEmpty]
    *** GOALS ***[handEmpty]
    **handEmpty
    pick up B from the table
    Place B on C
    [clear A, ontable A, ontable C, B on C, clear B, handEmpty]
    *** GOALS ***[A on B]
    **A on B
    Place A on B
    *** GOALS ***[clear B, holding A]
    **clear B
    [clear A, ontable A, ontable C, B on C, clear B, handEmpty]
    *** GOALS ***[holding A]
    **holding A
    pick up A from the table
    *** GOALS ***[ontable A, clear A, handEmpty]
    **ontable A
    [clear A, ontable A, ontable C, B on C, clear B, handEmpty]
    *** GOALS ***[clear A, handEmpty]
    **clear A
    [clear A, ontable A, ontable C, B on C, clear B, handEmpty]
    *** GOALS ***[handEmpty]
    **handEmpty
    pick up A from the table
    Place A on B
    ***** This is a plan! *****
    pick up B from the table
    Place B on C
    pick up A from the table
    Place A on B
\end{lstlisting}
まず,3・4行目で属性によってゴールが表現されていること示している.
その次に,5~7行目でこれらのゴールを属性からブロック名による表現に変換している.
続いて,8~14行目で属性によって式状態が表現されていることを示している.
先ほどと同様に,15~22行目では,これらの表現を属性によるものからブロック名のみの形に変更している.
したがって,以降の実行結果はブロック名のみを用いた場合と同様となる.

\subsection{考察}
属性を扱うに当たって,初めは属性のまま推論を行い,プランニングを実行できるようにしようと考えた.
しかし,同じブロックの属性であっても,異なる属性同士において,字面だけではマッチングが成功しないことに気が付いた.
例えば"A is blue"と"A is box"がルールとして与えられていたとする.
この時,ゴールが"blue on green"で,状態として"box on green"が保持されているものとする.
本来,blueとboxは同じブロックAの属性であるが,字面の上では,異なる定数として認識される. \\
これらを同一のブロックAを指し示すものとして認識するには,属性からブロックそのものへと変換する機能が必要である.
したがって,変換後のゴール"A on B"と状態"A on B"であればマッチングは成功する. \\
ここで,もう一度属性のまま推論を行う意義を考えることとした.
上記の方法では,マッチングにおいて最終的に属性からブロック名への変換処理を行っている.
ところが,初めから同様に属性をブロック名に変換する手法にすることでもマッチングは可能である.
例えば,あらかじめゴールを"blue on green"から"A on B"に変換し,初期状態を全てブロック名による表現にすることで,プランニング中の状態を"A on B"とする.
結果として,即座に"A on B"と"A on B"のマッチングが行われるようにする. \\
加えて,後者の方がゴールリストと初期状態のリストから属性表現をブロック名に変換するのみであるので,処理も非常にシンプルとなる.
後ろ向き推論の特徴と処理の簡潔さから,後者の手法を採用することにした. \\
複数ある実現方法の中から,既存のプログラムとの相性や論理的な汎用性を考えてもっとも良いと思える方法を選択できたと考えている.


\section{必須課題5-4}
\begin{screen}
    上記5-2, 5-3で改良したプランニングシステムのGUIを実装せよ.\par
    ブロック操作の過程をグラフィカルに可視化し,初期状態や目標状態をGUI上で変更できることが望ましい.
\end{screen}
私の担当箇所は,得られた結果を基にしたGUI本体の実装である.

\subsection{手法}
GUIで求められる出力が行えるように,以下のような機能を加えた.

\begin{itemize}
\item ブロック操作の過程を示す
\item ブロック操作の過程をグラフィカルに可視化する
\item 初期状態と目標状態をGUI上から画面入力で変更できる
\item 属性の決定をGUI上から行える
\item 操作に必要なオペレータと属性のデータを表示する
\item 禁止制約によってブロック操作が停止した場合にそれを表示する
\end{itemize}

これらのプログラムを実装するためのデータ受け渡し部分は青山君が作成してくれるため,ここでは受け取ったデータを基にしてどのような方法で表示を行うか考えながら実装を行った.実装はSwingによって行った.\\

また今回は初めてのGUI作成であったため,第一週では受け取るデータを既に取得したと仮定し,自ら作成した仮データを基にしてGUIについての検討を行った.第二週では実際にデータを受け取り,そのデータを基に正しく描画やプランニングが行えるようにプログラムを改良した.\\

\subsection{実装}
作成したGUIプログラムの中の主部分についての実装を以下に示す.\\

まず,GUIプログラムのmeinメソッドをソースコード\ref{main}に示す.
\begin{lstlisting}[caption=mainメソッド,label=main]
public static void main(String[] args){
	// 画像の定義
	images = new String[5][4];
	images[0][0] = "image/squBt.png";
	・・・
	images[4][3] = "image/daiDeft.png";

	// 対応番号の指定(色・形それぞれで定義)
	imageMapC = new HashMap<>();
	imageMapC.put("Blue", 0);
	・・・
	imageMapC.put("Default", 4);
	imageMapS = new HashMap<>();
	・・・

	// 属性指定での入出力に対応(defaultは予め弾く)
	Attribution = new HashMap<>();
	Attribution.put("Blue", 1);
	・・・
	Attribution.put("Yellow", 1);
	Attribution.put("box", 2);
	・・・
	Attribution.put("trapezoid", 2);

	PGUI frame = new PGUI();
	frame.setDefaultCloseOperation(JFrame.EXIT_ON_CLOSE);
	frame.setBounds(10, 10, 650, 450);
	frame.setTitle("プランニングシステム");
	frame.setVisible(true);
}
\end{lstlisting}
ここでフレームの大きさと表示位置・タイトルや,フレームを閉じると同時にプログラムを終了させるなどの動作を定めた.さらに,プログラム中で使用する変数であらかじめ値を設定しておくものの定義を行った.\\

次に,GUIプログラムのコンストラクタをソースコード\ref{PGUI}に示す.
\begin{lstlisting}[caption=コンストラクタ,label=PGUI]
PGUI(){
	// プレゼンターとの連結
	presenter = new Presenter();
	// 結果の格納(メソッド呼び出し)
	ArrayList<String> result = presenter.getPlan();
	results = new ArrayList<>(result);
	// 初期状態の追加
	results.add(0, "default position");
	// 結果ステップデータの取得
	pUR = presenter.getStepList();
	// 入力デフォルト値の格納(メソッド呼び出し)
	initialState = presenter.getInitialState();
	// 出力デフォルト値の格納(メソッド呼び出し)
	goalList = presenter.getGoalList();
	// 初期状態の格納
	String[] initialName = {"A", "B", "C"};
	String[] initialAColor = {"Default", "Default", "Default"};
	String[] initialAShape = {"default", "default", "default"};
	for (int i = 0; i < initialName.length; i++) {
		modelName.addElement(new String(initialName[i]));
		modelColor.addElement(new String(initialAColor[i]));
		modelShape.addElement(new String(initialAShape[i]));
	}
	// 禁止制約の格納
	prohibitRules.add("box on pyramid");
	・・・
	prohibitRules.add("trapezoid on ball");

	// 2ページ目以降のカード作成用メソッド
	createResultPage(pUR);
	// ボタンの作成メソッド
	createButton();
	// 最終処理メソッド
	finishData();
}
\end{lstlisting}
PGUIコンストラクタでPresenterプログラムを起動し,プランニングを行った結果を取得する.また同時に初期状態をGUIで表示するため,これらの情報も格納した.\\

ここから,先に述べた各機能の実装について詳しく記述していく.\\
まず始めにブロック操作の過程を示すプログラムをソースコード\ref{katei}に示す.
\begin{lstlisting}[caption=過程表示,label=katei]
JPanel tostring2 = new JPanel();
tostring2.setLayout(new BoxLayout(tostring2, BoxLayout.PAGE_AXIS));
tostring2.setBackground(Color.WHITE);
ArrayList<String> printResult = presenter.getPlan();
tostring2.add(new JLabel("***** This is a plan! *****"));
for (String printR : printResult) {
	tostring2.add(new JLabel(printR));
}
JScrollPane scrollpane2 = new JScrollPane(tostring2);
scrollpane2.setPreferredSize(new Dimension(200, 310));
BevelBorder border2 = new BevelBorder(BevelBorder.LOWERED);
scrollpane2.setBorder(border2);
JPanel Plan = new JPanel();
Plan.setLayout(new BoxLayout(Plan, BoxLayout.PAGE_AXIS));
Plan.add(new JLabel("Plan "));
Plan.add(scrollpane2);
\end{lstlisting}
ブロック操作の過程を示すため,新しく作成したパネルに受け取ったリストデータをラベルとして加え,そのパネルをフレームに追加することで表示を行った.また経路が長くなったときを考え,必要に応じてスクロールバーを追加できるようにした.\\

次にブロック操作の過程をグラフィカルに可視化するためのプログラムをソースコード\ref{graphic1},\ref{graphic2}に示す.このとき初期状態は別で取得するため,初期状態のみを描画するプログラムと,以降の過程を描画するプログラムを分けて作成した.
\begin{lstlisting}[caption=グラフィカル表示1(一部抜粋),label=graphic1]
// 変数決定,属性名の初期化,配置用・アーム用座標配列を定義
・・・
// 2ページ目の設定
JPanel page2 = new JPanel();
JLabel[][] p2Label = new JLabel[row][col];
GridLayout page2layout = new GridLayout();
page2layout.setRows(row); // 行数
page2layout.setColumns(col); // 列数
page2.setLayout(page2layout);
// テーブルの上に乗っているブロックの初期化
int next = 0;
for (String dataS : dataTable) {
	String[] state = dataS.split(" ", 0);
	for (int i = 0; i < blocks.size(); i++) {
		// 初めに入手したブロック名が何番目のものかcheck
		if (state[1].equals(blocks.get(i))) {
			// 名称一致のとき
			iconX[i] = row - 1;
			iconY[i] = next;
			next++;
		}
	}
}
// 他ブロックの上に乗っているブロックの初期化
for (String dataS : dataOn) {
	String[] state = dataS.split(" ", 0);
	for (int i = 0; i < blocks.size(); i++) {
		// それぞれの属性の番号を取得
		if (state[0].equals(blocks.get(i))) {
			ue = i;
		} else if (state[2].equals(blocks.get(i))) {
			sita = i;
		}
	}
	// 上部分の座標の確定
	iconX[ue] = iconX[sita] - 1;
	iconY[ue] = iconY[sita];
}
// blockとアームの上書き
for (int i = 0; i < blocks.size(); i++) {
	p2Label[iconX[i]][iconY[i]] = new JLabel(icon[i]);
	p2Label[iconX[i]][iconY[i]].setText(iconName[i]);
}
p2Label[armX][armY] = new JLabel(arm);
p2Label[armX][armY].setText(armname);
// アイコンの挿入
for (int i = 0; i < row; i++) {
	for (int j = 0; j < col; j++) {
		page2.add(p2Label[i][j]);
	}
}
\end{lstlisting}


\begin{lstlisting}[caption=グラフィカル表示2(一部抜粋),label=graphic2]
// 3ページ目以降
for (int i = 0; i < cardPage; i++) {
	// 初期化(2ページ目作成と同様の操作)
	・・・
	String hatenaX = pUR.get(i).getBindings().get("?x");
	String hatenaY = pUR.get(i).getBindings().get("?y");
	int hXz = blocks.indexOf(hatenaX);
	int hYz = blocks.indexOf(hatenaY);
	if (pUR.get(i).getName().equals("Place ?x on ?y")) {
		// xの操作
		iconX[hXz] = iconX[hYz] - 1;
		iconY[hXz] = iconY[hYz];
		// アームの操作
		armX = iconX[hXz] - 1;
		armY = iconY[hYz];
	} else if (pUR.get(i).getName().equals("remove ?x from on top ?y")) {
		// xの操作
		iconX[hXz] = iconX[hXz] - 1;
		iconY[hXz] = iconY[hYz];
		// アームの操作
		armX = iconX[hXz] - 1;
		armY = iconY[hYz];
	} else if (pUR.get(i).getName().equals("pick up ?x from the table")) {
		// xの操作
		iconX[hXz] = iconX[hXz] - 1;
		iconY[hXz] = iconY[hXz];
		// アームの操作
		armX = iconX[hXz] - 1;
		armY = iconY[hXz];
	} else if (pUR.get(i).getName().equals("put ?x down on the table")) {
		// xの操作,アームの操作
		iconX[hXz] = row - 1;
		armX = iconX[hXz] - 1;
		boolean umu;
		for (int j = 0; j < col; j++) {
			umu = true;
			for (int k = 0; k < iconY.length; k++) {
				if (j == iconY[k]) {
					umu = false;
					break;
				}
			}
			if (umu == true) {
				iconY[hXz] = j;
				armY = j;
				break;
			}
		}
	}
	// block・アームの上書き,アイコンの挿入
	・・・
}
\end{lstlisting}
上のように,初期状態は別のメソッドから受け取り予め描画を行い,そのデータを基にしてブロック操作の過程を描画した.また描画は座標で管理を行った.\\

さらに,初期状態と目標状態・各ブロックの属性をGUI上から変更,表示するためのプログラムをソースコード\ref{TextArea}と\ref{JList}に示す.
\begin{lstlisting}[caption=初期状態と目標状態の変更,label=TextArea]
// 手動入力用パネル
JPanel natural = new JPanel();
natural.setLayout(new BoxLayout(natural, BoxLayout.PAGE_AXIS));
// 入力(setInitialState)
JPanel sI = new JPanel();
iArea = new JTextArea(9, 20);
JScrollPane iScroll = new JScrollPane(iArea);
String ii = "";
for(String i : initialState) {
	ii += i + "\n";
}
iArea.setText(ii);
sI.add(iScroll);
// 入力(setGoal)
JPanel sG = new JPanel();
gArea = new JTextArea(4, 20);
JScrollPane gScroll = new JScrollPane(gArea);
String gg = "";
for(String g : goalList) {
	gg += g + "\n";
}
gArea.setText(gg);
sG.add(gScroll);
natural.add(sI);
natural.add(sG);
\end{lstlisting}
JTextAreaを用いて文字列の入力がGUI上で行えるようにした.また,入力文が増えた場合のためスクロールバーを必要に応じて表示した.\\

\begin{lstlisting}[caption=各ブロックの属性変更,label=JList]
JPanel allRadio = new JPanel();
// 色選択
JPanel p2 = new JPanel();
radio = new JRadioButton[4];
radio[0] = new JRadioButton("Blue");
・・・
// ボタンのグループ化
ButtonGroup group = new ButtonGroup();
group.add(radio[0]);
・・・
p2.add(new JLabel("Select Color"));
p2.add(radio[0]);
・・・
// 形状選択(色選択と同様の操作)
・・・

// 属性編集用ボタンの作成
JPanel ADS = new JPanel();
ADS.add(new JLabel("new Name "));
newNameText = new JTextField(5);
ADS.add(newNameText);
JButton add = new JButton("追加");
add.addActionListener(this);
add.setActionCommand("addButton");
ADS.add(add);
・・・
allRadio.add(ADS);
allRadio.add(p2);

// 属性入力用パネルの作成
JPanel attribution = new JPanel();
JPanel p4 = new JPanel();
// 属性の決定用パネル
JPanel nameD = new JPanel();
nameD.add(new JLabel("Determine Attribution "));
p4.add(nameD);
JPanel attribute = new JPanel();
JPanel NAME = new JPanel();
・・・
// リストで実現
namelist = new JList(modelName);
JScrollPane namesp = new JScrollPane();
namesp.getViewport().setView(namelist);
NAME.add(namesp);
attribute.add(NAME);
// 色選択リストパネル
colorlist = new JList(modelColor);
JScrollPane colorsp = new JScrollPane();
colorsp.getViewport().setView(colorlist);
// リストを選択不可にする
colorlist.setEnabled(false);
colorsp.setBorder(borderC);
JPanel COLOR = new JPanel();
・・・
COLOR.add(colorsp);
attribute.add(COLOR);
// 形状選択リストパネル(色選択と同様の操作)
・・・
p4.add(attribute);
\end{lstlisting}
今回の実装では色と形状は任意に選択することが出来ないため,これらをラジオボタンによりユーザーが選択するようにした.さらに,これらのデータの追加・削除・編集も行えるよう,属性データはJListを用いて管理を行った.また,新しい属性名の追加のためテキストフィールドを作成した.\\

また,操作に必要なオペレータを表示するプログラムをソースコード\ref{operator}に示す.
\begin{lstlisting}[caption=オペレータ表示プログラム,label=operator]
JPanel tostring = new JPanel();
tostring.setLayout(new BoxLayout(tostring, BoxLayout.PAGE_AXIS));
for (Operator operator : operators) {
	tostring.add(new JLabel("●Operator" + i));
	tostring.add(new JLabel("NAME: " + operator.getName()));
	tostring.add(new JLabel("ADD: " + operator.getAddList()));
	tostring.add(new JLabel("DELETE: " + operator.getDeleteList()));
	}
}
JScrollPane scrollpane = new JScrollPane(tostring);
scrollpane.setBorder(border);
\end{lstlisting}

最後に,禁止制約によってブロック操作が停止した場合の処理を行うプログラムをソースコード\ref{prohibit}に示す.
\begin{lstlisting}[caption=ブロックの状態を表すプログラム,label=prohibit]
JPanel prohibit = new JPanel();
JPanel hosoku = new JPanel();
LineBorder inborder = new LineBorder(Color.red, 2);
TitledBorder border = new TitledBorder(inborder,"Warning!!", TitledBorder.LEFT, TitledBorder.TOP);
JLabel setumei = new JLabel("      This Goal is not allowed by ProhibitRules");
hosoku.add(setumei);
・・・
hosoku.setBorder(border);
JPanel kari = new JPanel();
kari.add(hosoku, BorderLayout.CENTER);
prohibit.add(kari);
// 禁止制約のパネルの表示
// 属性名
JPanel prohibit2_1 = new JPanel();
for (int i = 0; i < modelName.size(); i++) {
	StringBuilder buf_1 = new StringBuilder();
	buf_1.append((String)modelName.get(i));
	buf_1.append(" is ");
	・・・
	buf_1.append((String)modelShape.get(i));
	prohibit2_1.add( new JLabel( buf_1.toString()) );
}
JScrollPane scrollpane1 = new JScrollPane(prohibit2_1);
JPanel AN = new JPanel();
JPanel an = new JPanel();
an.add( new JLabel("Attribution ") );
・・・
AN.add(an);
AN.add(scrollpane1);
// 禁止制約と目標状態(属性名と同様の操作を行う)
・・・
// 目標状態・禁止制約・属性名の設定
JPanel details = new JPanel();
details.add(AN);
details.add(PR);
details.add(GL);
prohibit.add(details);
\end{lstlisting}
目標状態が禁止制約に当たる場合は,目標状態の実現が不可能であるため,その目標状態が禁止制約に当たることを知らせる必要がある.今回はそのために新しくページを追加し,設定した属性名と禁止制約・目標状態を表示させ,この状態が禁止制約に当たることを知らせることでブロック操作が停止した場合の処理を行った.

\subsection{実行例}
作成したプログラムの実行結果を順に示す.\\
\clearpage
実行すると初めに初期状態・目標状態・属性の決定画面になる(図\ref{fig:page1}).デフォルトではABCの3つのブロックが存在し,その属性は全てデフォルト(黒色の四角形)で定義されている.\\

\begin{figure}[htbp]
  \begin{center}
    \includegraphics[scale=0.6]{images/page1.PNG}
    \caption{初期状態と目標状態,属性の決定画面}
    \label{fig:page1}
  \end{center}
\end{figure}
\clearpage
初めに実行を行った段階で,デフォルト状態でのプランニングが行われているため,ページ上部のNextボタンを押すと次の画面に遷移し,ブロック操作の過程の描画画面となる.Nextボタンを押すと次の状態へ,Prevボタンを押すと一つ前の状態へ遷移する様子が確認できる.(図\ref{fig:page2},図\ref{fig:page3},図\ref{fig:page4}).\\
%\clearpage
\begin{figure}[htbp]
  \begin{center}
    \includegraphics[scale=0.6]{images/page2.PNG}
    \caption{ブロック操作の過程1}
    \label{fig:page2}
  \end{center}
\end{figure}
%\clearpage
\begin{figure}[htbp]
  \begin{center}
    \includegraphics[scale=0.6]{images/page3.PNG}
    \caption{ブロック操作の過程2}
    \label{fig:page3}
  \end{center}
\end{figure}
%\clearpage
\begin{figure}[htbp]
  \begin{center}
    \includegraphics[scale=0.6]{images/page4.PNG}
    \caption{ブロック操作の過程3}
    \label{fig:page4}
  \end{center}
\end{figure}
\clearpage
さらに,Firstボタンを押すと一番初めの設定画面へ,Lastボタンを押すと一番最後の画面であるオペレータとプランニング結果を表す画面へと遷移する(図\ref{fig:page5}).\\
\begin{figure}[htbp]
  \begin{center}
    \includegraphics[scale=0.6]{images/page5.PNG}
    \caption{オペレータとプランニング結果の情報}
    \label{fig:page5}
  \end{center}
\end{figure}

\clearpage
次に,初期状態,目標状態,属性情報を変更してプランニングを行う.まず初期画面の左上に現在の属性情報があり,これを左下のラジオボタンを追加・削除・編集ボタンで変更する(図\ref{fig:page6}).\\

\begin{figure}[htbp]
  \begin{center}
    \includegraphics[scale=0.6]{images/page6.PNG}
    \caption{属性情報の変更}
    \label{fig:page6}
  \end{center}
\end{figure}
また,ブロック追加時には左下にある'new\ Name'欄に新規ブロックの名前を入力する必要があるが,無記入で追加をした場合や既にある名前を入力した際にはエラーメッセージがダイアログウィンドウで表示されるようになっている(図\ref{fig:page7},図\ref{fig:page8}).編集・削除ボタンを押した際に属性リストの名前を選択していなかった場合にも同様にエラーメッセージが表示される.\\
\begin{figure}[htbp]
  \begin{center}
    \includegraphics[scale=0.6]{images/page7.PNG}
    \caption{エラーメッセージの表示1}
    \label{fig:page7}
  \end{center}
\end{figure}
%\clearpage
\begin{figure}[htbp]
  \begin{center}
    \includegraphics[scale=0.6]{images/page8.PNG}
    \caption{エラーメッセージの表示2}
    \label{fig:page8}
  \end{center}
\end{figure}
\clearpage
初期状態と目標状態の変更は初期画面の右側のテキストボックスに記述する形で行う.このとき,範囲を超えた記述をする場合はスクロールバーが表示される(図\ref{fig:page9}).また,属性情報変更時と同様に初期状態と目標状態に何も記入せずPlanningボタンを押すとエラーメッセージが表示される.\\
%\clearpage
\begin{figure}[htbp]
  \begin{center}
    \includegraphics[scale=0.6]{images/page9.PNG}
    \caption{初期状態と目標状態の変更}
    \label{fig:page9}
  \end{center}
\end{figure}
\clearpage

変更を行った状態でPlanningボタンを押すと,定義した状態を基にプランニングが行われる.すると,探索が完了したことを伝えるメッセージが表示され,ブロック操作の過程の画面が新しく得られた結果が表示される.今回は例としてブロックを4つに増やし,属性情報を変えた(図\ref{fig:page10},図\ref{fig:page11},図\ref{fig:page12}).\\

\begin{figure}[htbp]
  \begin{center}
    \includegraphics[scale=0.6]{images/page10.PNG}
    \caption{再実行時のブロック操作の過程1}
    \label{fig:page10}
  \end{center}
\end{figure}
%\clearpage
\begin{figure}[htbp]
  \begin{center}
    \includegraphics[scale=0.6]{images/page11.PNG}
    \caption{再実行時のブロック操作の過程2}
    \label{fig:page11}
  \end{center}
\end{figure}
\begin{figure}[htbp]
  \begin{center}
    \includegraphics[scale=0.6]{images/page12.PNG}
    \caption{再実行時のブロック操作の過程3}
    \label{fig:page12}
  \end{center}
\end{figure}

\clearpage
またブロックを積み上げる動作だけでなく,ブロックをおろす場合もグラフィカルな可視化を行うことが出来る(図\ref{fig:page14},図\ref{fig:page16}).
\begin{figure}[htbp]
  \begin{center}
    \includegraphics[scale=0.49]{images/page14.PNG}
    \caption{積み下ろしのブロック操作の過程1}
    \label{fig:page14}
  \end{center}
\end{figure}
%\begin{figure}[htbp]
%  \begin{center}
%    \includegraphics[scale=0.6]{images/page15.PNG}
%    \caption{積み下ろしのブロック操作の過程2}
%    \label{fig:page15}
%  \end{center}
%\end{figure}
%\clearpage
\begin{figure}[htbp]
  \begin{center}
    \includegraphics[scale=0.49]{images/page16.PNG}
    \caption{積み下ろしのブロック操作の過程2}
    \label{fig:page16}
  \end{center}
\end{figure}

\clearpage
次に,初期状態と目標状態が同一であった場合の動作を示す.初期状態を示す画面の後に以下の表示が現れ,既に目標状態にあることが示される(図\ref{fig:page17}).
\begin{figure}[htbp]
  \begin{center}
    \includegraphics[scale=0.6]{images/page17.PNG}
    \caption{初期状態と目標状態が同一の場合}
    \label{fig:page17}
  \end{center}
\end{figure}

さらに,禁止制約によってプランニングが行えなかった場合の動作を示す.Planningボタンを押すと,プランニングが行えなかったことを示すエラーメッセージが表示され,ユーザーが定めた属性情報・禁止制約・ユーザーが定めた目標状態が表示される(図\ref{fig:prohibit1},図\ref{fig:prohibit2}).\\

\begin{figure}[htbp]
  \begin{center}
    \includegraphics[scale=0.6]{images/prohibit1.PNG}
    \caption{禁止制約を示すエラーメッセージの表示}
    \label{fig:prohibit1}
  \end{center}
\end{figure}
\begin{figure}[htbp]
  \begin{center}
    \includegraphics[scale=0.6]{images/prohibit2.PNG}
    \caption{属性情報・禁止制約・目標状態の表示}
    \label{fig:prohibit2}
  \end{center}
\end{figure}

\clearpage
\subsection{考察}
今回は新しい知識について学ぶ必要があったため,まず初めにプログラムを二週間という期間の中でどのように実装するかの計画を出来るだけ細かく立てた.これによっていつまでに何を終わらせておけば良いかが明確になり,スケジュール調整がし易くなったため,初めに分かる範囲で詳細な計画を立てることが重要であると分かった.\\

今回は状態遷移をグラフィカルに可視化するため,どのようにそれを描画するかがまず大きな課題であった.初めは一枚のフレームに状態遷移を全て表示することを考えていたが,それでは状態遷移が何回行われるか分からない今回の課題では表示方法に問題がある.そこで異なる方法を検討した.GUIについて調べる中で,CardLayoutを用いればカードの枚数を増やしていくことで状態の遷移回数の変化にも対応できると分かり,この実装でプログラムを考えた.\\

次に描画をどのように行うかについて考えた.GUI上での図形の描画は座標指定で行う必要がある.しかしこれではレイアウトマネージャーを無効にする必要がある.この場合描画を全て自身の座標管理で行わなければならないため,描画の難度が高いと考えた.そこで今回は図形を画像として予め作成しておき,これをImageIconとして取り込み,BoxLayoutで等分に分割した座標空間に対して割り当てていくことで実装を行うことを考えた.BoxLayoutを用いることで,空間の行と列を指定した数に等分に分割し,それぞれに画像を割り当てることが出来る.この方法を用いることで,座標指定をすることなく図形を配置でき,比較的容易に状態遷移を描画出来ると考えた.\par
しかしこの方法を用いると,取り込んだ図に対して操作を行うことは出来ないことが分かった.そのため,画像の上に重なるように図形の名称(A,B,C等)を入れることや,取り込んだ画像の色を上書きすることが出来ないため,今回は図形の横に名称を記載し,画像は色違いのものをはじめから用意しておくことでこの問題を解決した.\par
予め図形を作成しておくことから,図形や色の任意指定も出来なくなってしまった.しかし座標指定を行う場合も図形の任意指定はできないと考えられるため,図形の任意指定を行えるようにするためには異なる方法を検討する必要があると感じた.\\

また,表示範囲は図形の大きさと個数から定めることで,ブロックの個数の変化に対応できるようにした.この方法を用いることで,表示する個数が変化しても描画内容に大きな変化が起きないようになった.しかし初めに設定したフレームの大きさの関係で,5個以上のブロックの描画を行うと規定のフレームサイズを描画が超えてしまう.これにより,フレームをユーザー自身が大きくする必要が出てしまった.これについてはスクロールバーをつけることで対応しようと考えたが,実装が難しく実現できなかった.そのため描画は正しく行えたものの,表示方法に課題が残る結果となった.同時に,初めに作成するフレームのサイズも考えて決める必要があると分かった.\\

さらに属性情報の更新方法をどのようにするか考えた.属性情報では,名前・色・形を連動させ変更しなければならない.加えて任意での色や形の設定が出来ないため,これをラジオボタンで操作することとした.ここで項目を選択して編集を行えるとより使いやすくなると考え,JListを用いることを考えた.さらに名前・色・形をそれぞれ別のJListで表示することで,見やすい表示を心掛けた.\par
しかし名前・色・形をそれぞれJListで管理すると,それぞれの項目を選択出来てしまうため,ユーザーが現在どの項目を編集しようとしているのかが分かりにくくなってしまう.これを解決するため色と形のJListは,プログラム側で編集を行う時以外は無効にすることで,ユーザーが操作できないようにすることを考えた.\\

また今回はユーザー視点の操作について考えた実装を心掛けた.\par
まず,ユーザーが任意で記述できる初期状態や目標状態・属性情報は記述スペースを超えた場合はスクロールバーが表示されるようにした.また出力される経路についても,表示画面よりも長くなってしまった場合はスクロールバーでの表示が出来るようにした.さらに状態遷移の描画を行う際の経路の表示では,自身がどの状態にあるのかを分かりやすくするため,現在の状態を表す一文に背景色としてオレンジ色を付けた.\par
経路が長くなってしまった場合はスクロールバーが表示されるものの,そのままでは常にスクロールバーが一番上にある状態でページが作成される.これでは表示範囲を超えた状態遷移のページでは,自身の現在の状態を確認するためにはスクロールバーを毎回下げる必要がある.これではユーザーにとって使い難いGUIだと考え,常に現在の状態が見える位置に来るように表示する座標位置を順に下げ,ユーザーがスクロールバーを動かさずに現在の状態を確認できるよう実装を行った.\par
さらに禁止されている入力を行った場合などはメッセージを表示することで,何が問題でどのようにすれば良いのかを分かりやすく示せるようにした.


\section{発展課題5-5}
\begin{screen}
    ブロックワールド内における物理的制約条件をルールとして表現せよ.
    例えば,三角錐(pyramid)の上には他のブロックを乗せられない等,その世界における物理的な制約を実現せよ.
\end{screen}
\subsection{手法}
ブロックワールド内における物理的制約条件(以下、禁止制約と表現する)をルールとして表現し,プランニングに反映するために,
課題5-3で作成した属性に関する処理を扱うクラスAttributions内に禁止制約のルールを保持させる方法を考案した.
例えば,"box on pyramid"(三角錐の上に正四面体がある)のようなルールが物理法則に反する禁止制約に当たる. \\
これらの禁止制約のルール集合を,課題5-3で結びつけた「属性とブロックの関係」をもとに,属性を用いた表現からブロックによる表現に置き換える. \\
以上により,ゴールリストやオペレーター適用後の状態とブロックによって表現される禁止制約を比較することで,
物理法則に反する状態を削除することができる.
また,ゴールリストに物理法則に反するものが含まれる場合には,ゴール不成立とする. 

以下に,今回実装した禁止制約の例を示す. \\

\begin{figure}[!hbt]
  \centering
  \includegraphics[scale=0.35]{images/prohibit-rules.pdf}
  \caption{物理制約によって禁止される制約の例}
\end{figure}

\subsection{実装}
まず,属性を扱うAttributionsクラス内にArrayList<String>型の禁止制約のルール集合prohibitRulesを保持させる.
このprohibitRulesはコンストラクタから呼び出されるメソッドaddProhibitRules()により禁止制約のルールが追加される. \\
加えて,属性によって表現された禁止制約のルールをブロックによる表現に置き換えたルール集合prohibitBlockStatesも定義する.
prohibitRulesからprohibitBlockStatesに変換を行うに当たっては,課題5-3で作成したメソッドeditStatementListを用いる. \\
次に,状態と禁止制約を比較するメソッドcheckProhibitBlockState(String state)を実装した.
このメソッドを以下に示す.
\begin{lstlisting}[caption=checkProhibitBlockStateメソッドの実装, label=mid]
    private Boolean checkProhibitBlockState(String state) {
		for(String prohibitBlockState: prohibitBlockStates) {
			if(prohibitBlockState.equals(state)) {
				System.out.println("【Warning!:状態"+state+"は禁止制約です!!】");
				return false;
			}
		}
		return true;
	}
\end{lstlisting}
2~7行目に渡って,全ての禁止制約のルール集合の要素と仮引数で受け取った状態を比較する処理を行なっている.
ここで,一つでも禁止制約と一致すると,falseが返却される.
反対に,全ての禁止制約と一致しなかった場合には,trueが返却される. \\
続いて,このcheckProhibitBlockStateメソッドを利用するcheckStates(ArrayList<String> states)メソッドの説明を行う.
このメソッドについても以下に示す.
\begin{lstlisting}[caption=checkStatesメソッドの実装, label=mid]
    ArrayList<String> checkStates(ArrayList<String> states) {
		ArrayList<String> checkedStates = new ArrayList<String>();
		for(String state: states) {
			if(checkProhibitBlockState(state)) {
				checkedStates.add(state);
			}
		}
		return checkedStates;
	}
\end{lstlisting}
3~7行目に渡って,仮引数で受け取った状態集合の各要素をcheckProhibitBlockStateの引数として渡すことで,禁止制約のチェックを行なっている.
結果としてtrueが帰ってきた場合にはチェック済みの状態集合checkedStatesに格納し,falseの場合には格納されない. \\
全ての要素のチェックが終わると,checkedStatesを返却する. \\
checkedStatesメソッドを,変化直後の状態集合に対して使用することによって,物理法則に反する状態を削除することができる.
以上の実装により,禁止制約のブロックワールドへの反映を実現した.

\subsection{実行例}
以下に,CUI上での実行結果を示す.
ただし,コンストラクタはデフォルトコンストラクタを使用するものとする.
\begin{lstlisting}[caption=禁止制約によりゴールが成立しなくなった場合, label=mid]
    ~/Programming2/Work5
    ●java Planner                                                                                                                                                                                                    【 feature/prohibit-rules 】
    ###### Add prohibitRule ######
    ****** ProhibitRule:box on pyramid ******
    ****** ProhibitRule:box on ball ******
    ****** ProhibitRule:ball on pyramid ******
    ****** ProhibitRule:pyramid on ball ******
    ****** ProhibitRule:trapezoid on pyramid ******
    ****** ProhibitRule:trapezoid on ball ******
    ++++++ EditStatement ++++++
    box on pyramid =====> A on B
    box on ball =====> A on C
    ball on pyramid =====> C on B
    pyramid on ball =====> B on C
    trapezoid on pyramid =====> trapezoid on B
    trapezoid on ball =====> trapezoid on C
    ========== goal:green on ball ==========
    ========== goal:blue on pyramid ==========
    ++++++ EditStatement ++++++
    green on ball =====> B on C
    blue on pyramid =====> A on B
    【Warning!:状態B on Cは禁止制約です!!】
    【Warning!:状態A on Bは禁止制約です!!】
    ---------- initInitialState:clear blue ----------
    ---------- initInitialState:clear green ----------
    ---------- initInitialState:clear red ----------
    ---------- initInitialState:ontable box ----------
    ---------- initInitialState:ontable pyramid ----------
    ---------- initInitialState:ontable ball ----------
    ---------- initInitialState:handEmpty ----------
    ++++++ EditStatement ++++++
    clear blue =====> clear A
    clear green =====> clear B
    clear red =====> clear C
    ontable box =====> ontable A
    ontable pyramid =====> ontable B
    ontable ball =====> ontable C
    handEmpty =====> handEmpty
    禁止制約によってゴールが成立しなくなりました
\end{lstlisting}
はじめに,3~9行目で禁止制約のルールがprohibitRulesに追加されている.
次に,10~16行目でeditStatementListによって属性による表現からブロックによる表現に変換されている.
17,18行目で属性表現によって指定されたゴールが19~21行目でブロックによる表現に変換されている. \\
そして,22,23行目で禁止制約がゴールに使用されていることが検出されている.
詳しく見てみると,22行目の検出は14行目と20行目の変換結果が一致することによるものであり,
23行目の検出は11行目と21行目の変換結果が一致することによるものである.
これらによって,ゴールリストに物理法則に反するものが含まれていることが判明したため,39行目においてメッセージが表示され,プランニングが終了している.

\subsection{考察}
課題5-3の実装をもとに禁止制約をブロックワールドに反映する実装を行なった.
特に,editStatementListのように再利用することのできるメソッドを作成したことは大きなプラスとなった.
また,課題5-3で取り決めた,属性をブロック名に置き換えてからプランニングを行う方針は,
課題5-5を進めるに当たって大きな助けにもなった.
具体的には,"A is red","A is pyramid","B is blue","B is ball"のようなルールのもと,
"pyramid on ball"のような禁止制約があるとする.
このとき,"red on blue"が状態に含まれるとき,属性表現のままプランニングを行うと見逃してしまうことになるが,
先に状態と禁止制約の属性表現をブロック名に変換して,"A on B"とすることによって,
物理法則に反する状態であると判定することができる. \\
したがって,課題5-3で取り決めた方針によって,課題5-5のプログラムの質を向上させることができたと考えている. \\
以上より,より良い実装を目指すに当たって,あらかじめ頭の中でプログラムの構造や処理の手順を考えておくことは非常に効果的であると感じた.


\section{発展課題5-6}
\begin{screen}
    ユーザが自然言語(日本語や英語など)の命令文によってブロックを操作したり,初期状態/目標状態を変更したりできるようにせよ.
    なお,命令文の動詞や語尾を1つの表現に決め打ちするのではなく,多様な表現を許容できることが望ましい.
\end{screen}

\section{発展課題5-7}
\begin{screen}
    3次元空間 (実世界) の物理的な挙動を考慮したブロックワールドにおけるプランニングを実現せよ.
    なお,物理エンジン等を利用する場合,Java以外の言語のフレームワークを使って実現しても構わない.
\end{screen}


\section{発展課題5-8}
\begin{screen}
    教科書3.3節のプランニング手法を応用できそうなブロック操作以外のタスクをグループで話し合い,新たなプランニング課題を自由に設定せよ.
    さらに,もし可能であれば,その自己設定課題を解くプランニングシステムを実装せよ.
\end{screen}


% 参考文献
\begin{thebibliography}{99}
\bibitem{notty} Javaによる知能プログラミング入門 --著:新谷 虎松 \\
\bibitem{Swing} Let'sプログラミング Swingを使ってみよう, \url{https://www.javadrive.jp/tutorial/} (2019年12月8日アクセス).
\end{thebibliography}

\end{document}